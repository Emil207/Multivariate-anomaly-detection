%%% Research Diary - Entry

\documentclass[11pt,letterpaper]{article}

\newcommand{\workingDate}{\textsc{2019 $|$ October $|$ Week 42}}


\usepackage{../Misc/diary}

\begin{document}
\univlogo

\title{Research Diary}

{\Huge Weekly summary}\\[5mm]

\section*{Monday, October 14}
\textbf{AM:}

\textit{Distributed Keras/Tensorflow with Spark. } \newline
\url{http://web.stanford.edu/class/cs20si/lectures/march9guestlecture.pdf} \newline
Alternatives for distributed work:
\begin{itemize}
\item Uber's horovod
\item Estimator API
\item Dist-keras
\end{itemize}

\textbf{PM:}
\textit{Eager execution.}
\url{https://www.tensorflow.org/guide/eager} \newline
TensorFlow's eager execution is an imperative programming environment that evaluates operations immediately, without building graphs: operations return concrete values instead of constructing a computational graph to run later. This makes it easy to get started with TensorFlow and debug models, and it reduces boilerplate as well.




\section*{Tuesday, October 15}


\section*{Wednesday, October 16}


\section*{Thursday, October 17}
 

\section*{Friday, October 18}


\printbibliography

\end{document}