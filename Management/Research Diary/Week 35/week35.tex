%%% Research Diary - Entry

\documentclass[11pt,letterpaper]{article}

\newcommand{\workingDate}{\textsc{2019 $|$ August $|$ Week 35}}


\usepackage{Misc/diary}

\begin{document}
\univlogo

\title{Research Diary - Example Entry}

{\Huge Weekly summary}\\[5mm]

\section*{Monday, August 26}

Data sources: \newline
\url{http://odds.cs.stonybrook.edu/#table3}
\url{https://github.com/Microsoft/TagAnomaly/blob/master/data/dow_jones_index.csv}
\url{https://github.com/Microsoft/TagAnomaly/blob/master/data/sample-1H.csv}
\url{https://yahooresearch.tumblr.com/post/114590420346/a-benchmark-dataset-for-time-series-anomaly}

Started to work with the real traffic data from NAB. However this seems to be univariate as well. 

\section*{Tuesday, August 27}

Meeting with Maria:

Management:
Research diary. Should I send research diary?
Timeline. Tips for making it work in research work?

Report:
Report layout. Something to think about? 
Started to put notes into introduction.

Coding:
Data. Availability of MTS-data with labelled anomalies?
Numenta Anomaly Benchmark.
Data from Yahoo.

Probability of using different methods on different parts of the data. Wait with coding until I know which data I could use. 

\section*{Wednesday, August 28}

Organized the methods in the Methods chapter of the report. VARIMA seems unlikely to use since the high complexity of matrix inversion. Either take it down to less dimensions using PCA or use other techniques to reduce complexity. 

\section*{Thursday, August 29}

Plan: 
Start to study the methods. Understand, describe and mention pros and cons.
Send an email to Andreas.

\section*{Friday, August 30}
SM!

\printbibliography

\end{document}
