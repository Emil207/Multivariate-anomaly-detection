%%% Research Diary - Entry

\documentclass[11pt,letterpaper]{article}

\newcommand{\workingDate}{\textsc{2019 $|$ August $|$ Week 34}}


\usepackage{Misc/diary}

\begin{document}
\univlogo

\title{Research Diary - Example Entry}

{\Huge Weekly summary}\\[5mm]

\section*{Monday, August 19}
Worked at Sigma from 9AM to 5PM. I had a meeting with Johan and Markus regarding which data to use and how to progress. The following data is similar to the data that they usually encounter: \newline
\url{https://github.com/yzhao062/anomaly-detection-resources#31-multivariate-data} \newline
\url{https://github.com/khundman/telemanom}

Furthermore, I researched a GAN method for multivariate time series \cite{Li2019a}. This needs to be investigated further and an implementation made.

Started to make the main document in Overleaf and sync it with Mendeley.

\section*{Tuesday, August 20}

Arrived at office at 9 and attended a meeting with the group. Before lunch I continued with making the Overleaf mechanism automated, I developed the note taking mechanism and synched Overleaf with Github for backup.   

The note taking mechanism made will work like this. Weekly reports should be filed in PDF format while writing a daily report in the end of each working day. Ideas and interesting articles found can be linked to. If updates have to be made to the references.bib it should be moved to Misc in the Research Diary-folder manually.   
Evaluated different programs for project management, such as making a GANTT-scheme and using task management. I will use MS Project, although I had some problems getting it to work. Finally got it installed after reinstalling with Click-to-go instead of Windows Installer.

Found article about anomaly detection using extreme value analysis 
. Look at tomorrow!


\section*{Wednesday, August 21}

\section*{Thursday, August 22}

\section*{Friday, August 23}


\printbibliography 

\end{document}