%%% Research Diary - Entry

\documentclass[11pt,letterpaper]{article}

\newcommand{\workingDate}{\textsc{2019 $|$ August $|$ Week 34}}


\usepackage{Misc/diary}

\begin{document}
\univlogo

\title{Research Diary - Example Entry}

{\Huge Weekly summary}\\[5mm]

\section*{Monday, August 19}
Worked at Sigma from 9AM to 5PM. I had a meeting with Johan and Markus regarding which data to use and how to progress. The following data is similar to the data that they usually encounter: \newline
\url{https://github.com/yzhao062/anomaly-detection-resources#31-multivariate-data} \newline
\url{https://github.com/khundman/telemanom}

Furthermore, I researched a GAN method for multivariate time series \cite{Li2019a}. This needs to be investigated further and an implementation made.

Started to make the main document in Overleaf and sync it with Mendeley.

\section*{Tuesday, August 20}

Arrived at office at 9 and attended a meeting with the group. Before lunch I continued with making the Overleaf mechanism automated, I developed the note taking mechanism and synched Overleaf with Github for backup.   

The note taking mechanism made will work like this. Weekly reports should be filed in PDF format while writing a daily report in the end of each working day. Ideas and interesting articles found can be linked to.  
Evaluated different programs for project management, such as making a GANTT-scheme and using task management. I will MS Project or Excel, although I had some problems getting MS Project to work. Finally got it installed after reinstalling with Click-to-go instead of Windows Installer.

Found article about anomaly detection using extreme value analysis. Look at later! Read about version control.


\section*{Wednesday, August 21}

Started working at 12 after training in the morning. MS Project seems to deliver to much details and Excel will work fine for the timeline. Developed a timeline based on GANTT in Excel and started to fill it up.

Worked on goal document. Still have to decide how if GAN should be included and if time-frequency analysis is relevant. ARIMA is out for the moment since it will be too much work.

Plan:

Understand Git.
Sync Jupyter Notebook to GitHub.
Write in diary.

Deliver to Maria:
Goal Document
Data
Example of Layout


\section*{Thursday, August 22}

Plan:

Continue to explore the data.
Make a plan on how to structure the introduction.
Write in diary.

\section*{Friday, August 23}


\printbibliography 

\end{document}