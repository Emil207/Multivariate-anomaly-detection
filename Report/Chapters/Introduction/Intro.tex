%% Introduction %%
Anomaly detection, or outlier detection, in time series has historically been a common topic for research. Although this has mostly been with univariate data in smaller amounts. However, with the rise of the Internet of Things (IoT) and the rapid increase of computational power, this has changed. The amount of data is in many cases very big and in many cases the data is unlabeled.   A common origin of time series data are sensor systems that appear in many real-world applications such as space shuttles, health care monitoring and factories. In those systems, a lot of multivariate time series data is generated continuously, and it is often of relevance to find unexpected events. \cite{Malhotra2016a}

%% Applications %%
%Sensor data, mechanical systems diagnosis, medical data, network intrusion data, newswire posts or financial posts

%% Types of anomalies %%
In this thesis anomalies will be divided into either contextual or collective. Contextual anomalies refer to when values at specific time stamps suddenly change with respect to their temporally adjacent values. Collective anomalies refer to when entire time series or large subsequences within a time series have unusual shapes \cite{Aggarwal2013a}.